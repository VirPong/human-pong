\documentclass[12pt]{article}

\usepackage{hyperref}

% useful for formatting (align*, etc.) and for certain symbols (the QED box, etc.)
\usepackage{amsmath, amssymb, amsthm}

% for including graphics
\usepackage{graphicx}

% for conveniently specifying the spacing (\singlespacing, \doublespacing,
%    \onehalfspacing, etc.)
\usepackage{setspace}
\onehalfspacing

% this does some sort of symbol stuff
\usepackage{textcomp}

% A package for conveniently adjusting headers and such
\usepackage{fancyhdr}
\renewcommand{\headrulewidth}{0 pt}
\rhead{\textit{\thepage}}
\cfoot{}



% Set the margins
\usepackage[top=1.8cm, bottom=1.8cm, left=1.8cm, right=1.8cm]{geometry}

% Differently spaced itemize
\newenvironment{itemize*}%
  {\begin{itemize}%
  	\setlength{\parsep}{0pt}
    \setlength{\itemsep}{0pt}%
    \setlength{\parskip}{0pt}}%
  {\end{itemize}}
\newenvironment{enumerate*}%
  {\begin{enumerate}%
  	\setlength{\parsep}{0pt}
    \setlength{\itemsep}{0pt}%
    \setlength{\parskip}{0pt}}%
  {\end{enumerate}}


% set up a new command to insert a little bit of vertical space
% (use this BEFORE a line break)
\newcommand{\padding}{\vspace*{.5cm}}

% set up an environment to format each hw problem in
\newenvironment{problem}[1]{\noindent\textbf{#1.}}{\vspace*{.5cm}}

\newenvironment{proof*}{\par\noindent{\bf Proof}\quad}
               {\quad\vrule height 8pt depth 0pt width 8pt\medskip\par}



\begin{document}

%Add in some nice looking pages
 \begin{titlepage}
    \vspace*{\fill}
    \begin{center}
      {\Huge Final Report: Android Development Team}\\[0.5cm]
      {\Large Jillian Andersen, Jordan Apele, David Ruhle, Kyle Wenholz}\\[0.4cm]
      \today
    \end{center}
    \vspace*{\fill}
  \end{titlepage}
  
\tableofcontents
\newpage

%Cover Page (include group name and team member names)
%Table of contents

\section{The Final Product}
\label{sec:finalProduct}
%Describe in detail the end product your department is producing, do not 
%    forget about documentation artifacts.

\subsection{Requirements}
\label{sec:requirements}
%Functional Requirements:
%    Use Cases:
%        All uses cases must use the same template and this template should 
%            identify Actors, Preconditions, Postconditions, Scenario, and 
%            Alternatives for fully-dress uses cases. The template should 
%            also include a meaningful name for the use-case and some form of
%            versioning.
%        Fully Dressed Descriptions: You need to write fully-dressed well 
%            detailed use case for all features you plan to include in your 
%            final product. 
%    System Sequence Diagram:
%        Create a system sequence diagram for your most important 
%        fully-dressed use case. Include a one paragraph description of what 
%        is being depicted in the diagram (use plan English). 
%Nonfunctional Requirements: Detail all nonfunctional requirements that you 
%    addressed in your final product (e.g. Easy Visibility: All text in our 
%    product is 16pt font.)

\section{Diagrammatic Depictions of the Product}
\label{sec:diagrams}
%Domain Analysis: Create a UML diagram that depicts the domain model for your
%    product (this will be a revision from Report 1). Include a plain-English
%    description of what is being shown in the diagram. Define any terms used
%    in conceptual classes, attributes, or associations that might not be 
%    clear to a lay person.


%Interaction Diagram:
%    Create an interaction diagram for each of the events depicted in your 
%    System Sequence Diagram.
%    Add a description of the design principles (Information Expert, Creator,
%    High Cohesion, Low Coupling, Controller) you employed, where you employed
%    them and why you made those choices. 

\section{Implementation}
\label{sec:implementation}
%Implementation:
%    Include the coding style guide that your team is using. This should be 
%    fairly detailed including naming, coding conventions, and comment 
%    conventions.
%    Install documentation: include a description of all necessary procedures 
%    a developer would have to complete to install your product. If you are 
%    assuming a certain starting environment then explicitly state so (e.g. a 
%    Linux server with Apache installed). Make sure to include how the user 
%    would access and download your source code and documentation. If you had 
%    difficulty installing from the previous teams description be sure to 
%    correct those difficulties.
%    Include a current class diagram of your product
%        Diagram should depict all classes and their associations 
%    Description of algorithms, data structures and design patterns
%        Describe any complex algorithms, data structures, or design patterns 
%        your group used. Provide insights as to why you made the choices you 
%        did.
%        Describe any techniques you are using to ensure fault tolerance (e.g.
%        if you have information to write to a db but the db is down what do 
%        you do?) 
%    Data Storage:
%        Identify all the data you are storing (ex. user athentication, 
%        medical records, back up information if the DB is down etc.)
%        If your product contains a database include both an ER Diagram and 
%        the schema for it, include a description of why you made the design 
%        decisions you did.
%        If the data is not stored in a db describe how it is stored included 
%        formatting. 
%    Describe your testing and verification procedure for your implemented 
%    code 

\section{Reflections}
\label{sec:reflections}
%Reflections:
%    Describe the technical challenges you encountered in the development of 
%    your product.
%    Describe how the software engineering techniques you learned in this 
%    course helped you in your development.
%    Describe what you would have done differently if you were to start this 
%    project over again.
%    If you were to continue this project what would you do. 



%References: Clearly indicate all the tools and sources you have used in the 
%    development of your product thus far. 

\newpage
\bibliographystyle{amsplain}
\bibliography{finalReport.android.bib}








\end{document}