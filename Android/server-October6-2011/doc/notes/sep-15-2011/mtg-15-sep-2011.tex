% !TEX TS-program = pdflatex
% !TEX encoding = UTF-8 Unicode

% This is a simple template for a LaTeX document using the "article" class.
% See "book", "report", "letter" for other types of document.

\documentclass[11pt]{article} % use larger type; default would be 10pt

\usepackage[utf8]{inputenc} % set input encoding (not needed with XeLaTeX)

%%% Examples of Article customizations
% These packages are optional, depending whether you want the features they provide.
% See the LaTeX Companion or other references for full information.

%%% PAGE DIMENSIONS
\usepackage{geometry} % to change the page dimensions
\geometry{letterpaper} % or letterpaper (US) or a5paper or....
% \geometry{margins=2in} % for example, change the margins to 2 inches all round
% \geometry{landscape} % set up the page for landscape
%   read geometry.pdf for detailed page layout information

%\usepackage{graphicx} % support the \includegraphics command and options

% \usepackage[parfill]{parskip} % Activate to begin paragraphs with an empty line rather than an indent

%%%% PACKAGES
%\usepackage{booktabs} % for much better looking tables
%\usepackage{array} % for better arrays (eg matrices) in maths
%\usepackage{paralist} % very flexible & customisable lists (eg. enumerate/itemize, etc.)
%\usepackage{verbatim} % adds environment for commenting out blocks of text & for better verbatim
%\usepackage{subfig} % make it possible to include more than one captioned figure/table in a single float
%% These packages are all incorporated in the memoir class to one degree or another...
%
%%%% HEADERS & FOOTERS
%\usepackage{fancyhdr} % This should be set AFTER setting up the page geometry
%\pagestyle{fancy} % options: empty , plain , fancy
%\renewcommand{\headrulewidth}{0pt} % customise the layout...
%\lhead{}\chead{}\rhead{}
%\lfoot{}\cfoot{\thepage}\rfoot{}
%
%%%% SECTION TITLE APPEARANCE
%\usepackage{sectsty}
%\allsectionsfont{\sffamily\mdseries\upshape} % (See the fntguide.pdf for font help)
%% (This matches ConTeXt defaults)
%
%%%% ToC (table of contents) APPEARANCE
%\usepackage[nottoc,notlof,notlot]{tocbibind} % Put the bibliography in the ToC
%\usepackage[titles,subfigure]{tocloft} % Alter the style of the Table of Contents
%\renewcommand{\cftsecfont}{\rmfamily\mdseries\upshape}
%\renewcommand{\cftsecpagefont}{\rmfamily\mdseries\upshape} % No bold!
%
%%%% END Article customizations

%%% The "real" document content comes below...

\title{Meeting Notes}
\author{Server Group}
\date{15 Sep 2011} % Activate to display a given date or no date (if empty),
         % otherwise the current date is printed 

\begin{document}
\maketitle

\section{Members Attended}

\begin{itemize}
	\item Shelby Lee
	\item Ryan Wheeler
	\item Patrick Green
	\item Daniel Guilak (M)
\end{itemize}

\section{Review of Goals}

\begin{itemize}
	\item Understanding how we lay out the database: still formulating a general plan
\end{itemize}

\section{What has been accomplished}

\begin{itemize}
	\item Figured out how database will be lay out – Daniel met with other managers and is still figuring out how a general plan.
       \item Good start on the deciding the scope of the server -- relaying information, calculating game dimensions, etc.
       \item It's possible to create servers in Java (and feasible) -- Ryan and Patrick are working on it.
       \item Decided against mySQL for more concurrent database system, and current database possibilies (Terrastore?) have good Java APIs.
       \item Server machine is set up and ready to go.
\end{itemize}

\section{New Goals}

\begin{itemize}
	\item Get everyone working on source control (all members comfortable)
       \item Install database on server: be able to write and read from database, getting server running; finalize components for server (users/password/real-time streaming/data store): determine all the jobs of the server; update wiki whenever we find anything out
       \item Patrick and Ryan: server end: open a port and be able to send data back and forth from a phone.
       \item Shelby: figure out Terrastore database implementation, downloading, and functions (if need help, contact other members)
       \item Ryan: look further into setting up servers with Java
\end{itemize}

\end{document}
