\documentclass[letterpaper,12pt]{article}

\begin{document}
\title{Initial Plan -- Server Group}
\author{Ryan Wheeler, Shelby Lee, Patrick Green, Daniel Guilak}
\date{\today}
\maketitle

\tableofcontents

\section{Team Profile}
\subsection{Ryan Wheeler}
	Ryan has experience in coding in many different languages and because of this he has a lot of experience in adapting to different languages and change his programming style while working on a project in order to complete those projects.  By learning many different programming languages he has gained experience in researching languages and how features of those languages work.  Because he has had experience researching and changing programming languages which makes him able to complete a project in a timely manner with a working product.  He also has a lot of experience working in small teams and working with them to meet deadlines.

	His qualifications for working at Vir-Pong are that he has two years of programming experience, experience with five different programming languages (Java, OpenGL, MIPS, Haskell, and Prolog) and four programming applications (BlueJ, Eclipse, Microsoft Visual Studio, and MARS).  At the University of Puget Sound he has taken six computer sciences classes including: Intro to Computer Science, Computer Science 2, Assembly Language and Computer Architecture, Computer Graphics, Mathematics and Computer Science, and Programming Language Paradigms.
\subsection{Shelby Lee}
Languages \& Software: BlueJ, Eclipse, HeidiSQL, Java, SQL
Operating Systems: Microsoft, Macintosh
Strengths: documentclassation, organization
Has taken computer science courses at the University of Puget Sound. Courses include: introduction to computer science, computer science II, mathematics of computer science, and databases.

	Her most relevant experience is constructing a bank simulation for class. Coded in group environments, the project focused on event-driven communication and continual unit and integration testing. This project exemplified Shelby’s group communication and documentation abilities.

	Furthermore, in her database course, Shelby researched the fundamental structure of document store databases. She was already familiar with the underlying functions and theoretical framework composing mongoDB.
\subsection{Patrick Green}
	Patrick Green has qualifications in Java programming. He knows how to use different IDEs such as BlueJ, Eclipse and JUnit. Patrick has strengths in testing code and troubleshooting. He is very qualified to work in team settings and in groups. Patrick can be reliable to show up for meetings and be ready to contribute to the team. Even with only one year of programming Java experience, he is determined to work harder to learn more computer science skills. He has taken Introduction to Computer Science, Computer Science 2 and Mathematics of Computer Science.
\subsection{Daniel Guilak}
Languages \& Software: Java, Android SDK, MATLAB, Python, BASH scripting, C/C++, \LaTeX.
Operating Systems: Linux, Windows, Macintosh.

Hardware: Built and configured numerous Linux and Windows computers.

Strengths: Programming, design, organization.

	Daniel is an avid open-source advocate and has interned at the IBM Linux Technology Center and Oregon Health and Sciences University Biomedical Engineering Department during high school. He is extremely excited to try his hand in a start-up software development company making products for consumers.

\section{End Product}
	The Server Team branch of the Vir-Pong project handles a majority of the communication in this project. The end product is a server that handles all communication from any browser, device running the Android platform, or Apple products running iOS. The server will handle game logic, such as ball movement and the game rules and communicate them effectively to active clients. The server will be able to take data from any connected devices and relay data to the opponent.The code that handles this data will be written in JavaScript. The server also will handle the communication needed by the Website Team and will provide them with information that they request from databases and other entities present on the server. The end product will contain very detailed documentation on where tools were downloaded from and how to replicate the project. The documentation will also include installation directions and links to the websites that were used. This server’s database that holds user names, passwords and important game information will run on MongoDB.
The server will accept connections from devices wishing to play pong against each other, and then after a brief handshake period, will accept each device's paddle location, and trasmit back the opponent's location, ball direction, position, and velocity, as wellas other relevant game information such as score. The server will keep a record of users and their win/loss ratios, and potentially previous game state information so that games could be replayed.
\section{Features}
\subsection{Key Features}
	The key features that our team needs to implement vary greatly in complexity.  The most basic feature of our server is for it to communicate with multiple clients.  Whether a client  is a phone or the website, for our project to be successful it has to be able to connect to both and send data and receive data to both.  Another basic property of the server is to contain a database that can store user names and passwords.  In order to give the virtual ping-pong community a way to come together and play virtual ping pong with each and create tournaments we have to give the users the ability to create their own unique accounts.  In addition to the holding the user data, the server will also be in charge of tracking much of the movement that takes place within the game and communicating that movement to the phones.  The first way that the server is responsible for movement is concerned with receiving paddle/player movement from each of the phones and relaying that information to both of the phones.  The other movement that the server is responsible for is tracking the movement of the ping pong ball and sending players an update of its progression.  We considered this model best because it would give not a player a host advantage making the player who is not hosting the game a disadvantage because that player would need to player host movement and ball movement which would have given the none host player some lag during the game.
\subsection{Secondary Features}
	The features that we consider secondary to the server are streaming games and real-time games.  Streaming games is a secondary feature because we feel that while it would be a worthwhile experience for those watching, make our virtual ping pong website unique, and possibly entice ESPN into covering virtual ping pong games, it would be no good to give them a lag-ridden and broken game.  If we get everything working perfectly with the game then adding a streaming service as a supplement to the great game will make our product even better.  As for what we mean by real time games it is our goal to receive and send the information that is passed to the server is as little time as possible (preferably in the nanoseconds or milliseconds), however, if it turns out that we are unable to do this then it would not hurt our product all that much if we had input lag that was a bad as one fourth of a second or even half a second.  Tournament features are also a secondary feature; they would better facilitate player-to-player interaction and include concepts of rank to their game history. 
\section{Tools}
\subsection{Git and Github}
	\begin{itemize}
		\item Git (http://git-scm.com/) is a distributed version control system that is widely-used in the open source community.
		\item Github (http://github.com/) is a web-based hosting service for projects using git. It provides collaboration tools and many other features such as pull requests and commit management that helps small development groups work efficiently.
		\item The current project repository is hosted on github and collaboration will happen through the web-based UI.
	\end{itemize}
\subsection{node.js}
	\begin{itemize}
		\item node.js (http://nodejs.org/) is an easy-to-learn server-side JavaScript server development environment that supports a myriad of different communication protocols such as HTTP, TCP, and WebSockets.
		\item Will be used as the framework for the server.
	\end{itemize}
\subsection{MongoDB}
	\begin{itemize}
		\item MongoDB (http://mongodb.org/) is a scalable open-source non-relational database with strong node.js support.
		\item Will be used as the main database for storing and retrieving game information.
	\end{itemize}
\subsection{Eclipse}
	\begin{itemize}
		\item Eclipse is an Integrated Development Environment written in Java that provides support for many different programming languages and version control systems through plugins -- A git plugin is available, called Egit (http://eclipse.org/egit/).
		\item Most team members will be using Eclipse IDE for development.
	\end{itemize}
\subsection{Google Docs}
	\begin{itemize}
		\item Google Docs (http://docs.google.com/) is an online document processor with tools similar to Microsoft Word, Powerpoint, and others. It allows for collaboration between different users concurrently on the same document.
		\item Will be used for collaboration on internal documents and documentation.
	\end{itemize}

\section{Challenges}
\subsection{Communication between The Server, Phones, and Websites}
	Communication is the fundamental purpose of the server. Due to this, we must be able to have clients connect and disconnect smoothly.
 
 	Our current plan to solve the issue of communication is working tightly with the phone teams and web development teams. We are attempting to standardize the tools we are using across our teams to allow for easier communication (such as Javascript). We also plan to rely heavily on the web interface to funnel user identification, thus helping to facilitate phone communication with the web’s server as well as the main server.
\subsection{Data Structure and Organization}
	One of the imminent concerns we face is determining exactly what data we will be storing and how we will be storing it. How we solve this problem determines both game-play performance as well as the stability of the server.
 
	In order to solve this issue, we will keep in constant communication with the Android, iPhone, and website teams to understand what type of data can be accurately communicated to and from the server with little error. We have roughly concluded that the server will handle the game mechanics (the ball tracking system) as well as real-time updates of paddle positions to both the phones and website.
	                 	
	We will determine the core-structure of our database in mongodb based on the key features. Thus, we plan to index the game database based on game-ID (possible a combination with user-ID) that handles and updates the constant stream of paddle and ball placement.        	
\subsection{Gameplay}
	From experience, all multiplayer online games experience lag and packet loss. We hope to prevent this as much as possible.
 
	One way is that we are using a non-relational database. Non-relational databases tend to have quicker input and access times than relational databases, making them better tools for online gaming servers. This may not be enough, and keeping the game running in real time may require a cap of games occurring at once if can find no other solution.
\section{Schedule}
\begin{itemize}
	\item Oct 7 -- A database implemented on the server.
	\begin{itemize}
		\item  Be able to have the server do a simple query from the mongoDB database and display it.
	\end{itemize}
	\item Oct 7 -- Get the server up and running. 
	\begin{itemize}
		\item Be able to have a server that provides text or visual response to simple sample packets from other computers and relays sample information from the database.
	\end{itemize}
	\item Oct 11 -- Basic server communication to other devices. 
	\begin{itemize}
		\item Have a proof of concept for devices on the Android and iOS platforms to be able to communicate with the server. 
		\item Have the clients be able to display information queried from the mongoDB database by communicating with the server.
	\end{itemize}
	\item Oct 17 -- Simple game engine/logic working.
	\begin{itemize}
		\item Translate the rules of two-player (non-computer opponents) pong into JavaScript code that the server can use.
		\item Most likely will test on a browser such as Chrome or Chromium first.
	\end{itemize}
	\item Nov 4 -- Real-time game streaming on web browsers.
	\begin{itemize}
		\item Be able to watch a game that is happening between two other clients on a third client (most likely a browser at first
	\end{itemize}
\end{itemize}
\end{document}
