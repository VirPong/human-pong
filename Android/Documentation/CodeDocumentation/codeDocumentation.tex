\documentclass[12pt]{article}

\usepackage{hyperref}

% useful for formatting (align*, etc.) and for certain symbols (the QED box, etc.)
\usepackage{amsmath, amssymb, amsthm}

% for including graphics
\usepackage{graphicx}

% for conveniently specifying the spacing (\singlespacing, \doublespacing,
%    \onehalfspacing, etc.)
\usepackage{setspace}
\onehalfspacing

% this does some sort of symbol stuff
\usepackage{textcomp}

% A package for conveniently adjusting headers and such
\usepackage{fancyhdr}
\renewcommand{\headrulewidth}{0 pt}
\rhead{\textit{\thepage}}
\cfoot{}



% Set the margins
\usepackage[top=1.8cm, bottom=1.8cm, left=1.8cm, right=1.8cm]{geometry}

% Differently spaced itemize
\newenvironment{itemize*}%
  {\begin{itemize}%
  	\setlength{\parsep}{0pt}
    \setlength{\itemsep}{0pt}%
    \setlength{\parskip}{0pt}}%
  {\end{itemize}}
\newenvironment{enumerate*}%
  {\begin{enumerate}%
  	\setlength{\parsep}{0pt}
    \setlength{\itemsep}{0pt}%
    \setlength{\parskip}{0pt}}%
  {\end{enumerate}}


% set up a new command to insert a little bit of vertical space
% (use this BEFORE a line break)
\newcommand{\padding}{\vspace*{.5cm}}

% set up an environment to format each hw problem in
\newenvironment{problem}[1]{\noindent\textbf{#1.}}{\vspace*{.5cm}}

\newenvironment{proof*}{\par\noindent{\bf Proof}\quad}
               {\quad\vrule height 8pt depth 0pt width 8pt\medskip\par}



\begin{document}

%Add in some nice looking pages
 \begin{titlepage}
    \vspace*{\fill}
    \begin{center}
      {\Huge Code Documentation: Android Development Team}\\[0.5cm]
      {\Large Jillian Andersen, Jordan Apele, David Ruhle, Kyle Wenholz}\\[0.4cm]
      \today
    \end{center}
    \vspace*{\fill}
  \end{titlepage}
  
\newpage


Due to the nature of web-development, not all of our code is ``documentation 
worth''.  There are, of course, comments throughout all of the code, but 
the pieces developed by our team that are worthy of a full document are below.

After collaborating with the iOS team for an entire semester, our team was 
in the end responsible for only two files: pong.js and replay.js.   The 
remainder of the code was mostly styling and, therefore, fell into their 
territory.  JavaScript has no official documentation procedure, so we 
have attempted to mimic JavaDoc comments as best as possible.
\singlespacing
\section{pong.js}
This file deals with the VirPong game: drawing, communicating (server and 
inputs), and all other necessary utilities.
\subsection{Drawing Code}

\begin{itemize*}
\item function initCanvas() \\
Starts the pong game and grabs the canvas so that we can modify it in JS.
\item function displaySelection(selection, options)\\
Serves as a filter for various display options.  Game canvases, buttons, input options, and even game end screens are initialized here.\\
 @param {selection} a string representing the desired display: gameCanvas, gameCanvasWithButtons, inputMethodSelection, selectRoom, newRoom, or gameEnd\\
 @param {options} any of various options that go with {selection}
\item function draw() \\
Draws the game state.
\item function drawHalfCourt()\\ 
Draws a half court line.
\item function drawPaddles()\\
Draws paddles based on the field positions.
\item function drawRect(a,b,c,d,col)\\
Draws rectangles on the canvas.\\
@param a top-left x-position\\
 @param b top-left y-position\\
 @param c bottom-right x-position\\
 @param d bottom-right y-position\\
 @param col color of the paddle
\item function drawBall()\\
Draws the ball at current position.
\item function drawScore()\\
Draws current score on the canvas.
\end{itemize*}

\subsection{Input Code}
\begin{itemize*}
\item function handleInputSelect(method)\\
Initializes the appropriate input methods for playing a game.\\
@param {method} a string for the desired input method. \\
"K" = Keyboard\\
"T" = Touchscreen\\
"W" = Wii Remote\\
"A" = Local Accelerometer
\item function movePaddle(e)\\
Recieve the input and send it to changePaddlePosition(), which actually changes the paddle position.\\
@param {e} the event passed by the keypress.
\item function changePaddlePosition(actualKey)\\
Change the value of leftPaddle or rightPaddle so that it will draw in the correct place.  This method is for a string input.  "W" is down and"S" is up. \\
@param {actualKey} The string value of the key pressed.
\item function detectViableInputMethods()\\
Figure out what platform we"re running on, return the correct choices for input selection. 
\item function setupLocalAccerometer()\\
Set the frequency of refresh for accelerometer data.  Start watching acceleration and point ot it with watchID.
\item function onSuccess(acceleration)\\
Contains the work done each time acceleration is audited. Right now, we display the raw data with a timestamp, as well as the calculated position, which is taken from the getPosition function.\\
@param   acceleration    An object containing the current acceleration values. (x,y,z,timestamp).
\item function onError(acceleration)\\
Fires off an alert if there's an error if the collection of acceleration.
\end{itemize*}

\subsection{Server Code}
\begin{itemize*}
\item document.addEventListener("DOMContentLoaded", function()\\
The 'document.addEventListener' contains reactions to information sent by the server.
\item function connectToServer()\\
Sets up the connection to the server and registers multiple event listeners.
\item joinRoom(room, clientType)\\
Select the room to join.\\
@param room a string representing the room name\\
@param clientType player or spectator as a string
\item function createRoom(roomName)\\
Tells the server to make a room just for me.\\
@param {roomName} the name of the room to create.
\item function updatePaddleToServer(position)\\
Update our paddle position with the server.\\
@param {position} the new position of the paddle.
\item function performAuthentication()\\
Loads in login information from local storage to send to the server.
\item handleNewRoom()\\
Grabs the room name from an element roomName.roomName and submits it as a new room to the server.
\end{itemize*}


\singlespacing
\section{replay.js}
This file takes care of backend events for the replay system (still in beta):
drawing and server communication.

\subsection{Drawing Code}

\begin{itemize*}
\item function initCanvas()\\
Stars the pong game and grabs the canvas so that we can modify it in JS.
\item function draw()\\
Draws the game state.
\item function drawHalfCourt() \\
Draws a half court line.
\item function drawPaddles()\\
Draws paddles based on the field positions.
\item function drawRect(a,b,c,d,col)\\
Draws rectangles on the canvas.\\
@param a top-left x-position\\
@param b top-left y-position\\
@param c bottom-right x-position\\
@param d bottom-right y-position\\
@param col color of the paddle
\item function drawBall()\\
Draws the ball at current position.
\item function drawScore()\\
Draws current score on the canvas.
\end{itemize*}

\subsection{Server Code}
\begin{itemize*}
\item document.addEventListener("DOMContentLoaded", function()\\
The 'document.addEventListener' contains reactions to information sent by the server.
\item function connectToServer()\\
Sets up the connection to the server and registers multiple event listeners.
\item function viewGame(gameName)\\
Select the room to join.\\
@param room a string representing the room name.\\
@param clientType player or spectator as a string.
\item function createRoom(roomName)\\
Tells the server to make a room just for me.\\
@param {roomName} the name of the room to create.
\item function updatePaddleToServer(position)\\
Update our paddle position with the server.\\
@param {position} the new position of the paddle.
\item function performAuthentication()\\
Loads in login information from local storage to send to the server.
\item function displaySelection(selection, options)\\
Serves as a filter for various display options.  Game canvases, buttons, input options, and even game end screens are initialized here.\\
@param {selection} a string representing the desired display: gameCanvas, gameCanvasWithButtons, inputMethodSelection, selectRoom, newRoom, or gameEnd\\
@param {options} any of various options that go with {selection}\\
\end{itemize*}




\end{document} 








